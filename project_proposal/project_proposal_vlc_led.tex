\documentclass{article}
\usepackage[a4paper, tmargin=1in, bmargin=1in]{geometry}
\usepackage[utf8]{inputenc}
\usepackage{graphicx}
\usepackage{parskip}
\usepackage{pdflscape}
\usepackage{listings}
\usepackage{hyperref}


\newcommand{\ra}{$\rightarrow$}


\title{EE 344 Project Proposal - Visible Light Communication using LED}
\author{
Arka Sadhu - 140070011\\
Sudeep Salgia - 14D070011\\
Parth Kothari - 14D070019\\
}
\date{November 2016}

\begin{document}
\maketitle

\tableofcontents
\newpage
\section{Abstract}
% Five to ten sentences providing a non-technical description; use technical words only to describe important specifications. Describe the “big picture” of the project.
LiFi (Light Fidelity) is the new buzzword in new era of communication. It is often used to describe high speed Visible Light Communication (VLC). VLC dedicates itself to solving the problem of illumination and communication together. The way it is implemented is that the light source is switched on and off at a very high speed, such that it is not discernible to the human eye, but a sensitive photodiode can detect it. This solves the problem of illumination and at the same we can encode information in the switching pattern and hence enabling us to transmit data, and recover it using the photodiode receiver.

In our project, we aim to develop a prototype of Visible Light Based Communication link. The project will consist of a transmitter, a LED in this case, which will transmit the message signal which will be received by the receiver. The Transmitter will transmit a file preloaded on a USB/SD card and this will be captured by receiver. 

\section{Project Description}
\subsection{Background and Motivation}
% related to your work, what others have done with current limitations, references, literature survey, or what is the pain in the current scenario.
% Electronic Design Lab is giving us a great opportunity to explore this domain and  all of us are enthusiastic to work on it , thus motivating us to take up this topic.
% ********************************************************************
% copied from : http://www.vlcc.net/modules/xpage0/?ml_lang=en
% **************** Need to edit regarding what is the pain in the current scenario**************************************
Visible Light Communication is the most advanced communication technology using "Visible Light"; the visible light everywhere around our daily life. We are heavily relying on our eyes to gather almost all information for our day-to-day activities. "Visibility" is one of the most important things for human being, and many devices are developed to assist our "Visibility". 

For instance, there are many devices including the lightings in our offices, home, the lightings on roads, traffic signals, commercial displays, small lamps on electronic home appliances including TVs, etc. 

Recently, LED (Light Emitting Diode) has been used for those devices. LED has a special characteristic to light on and off very fast. The data can be transmitted by lighting LED on and off at ultra high speed. And, the digital camera and the camera on cell phones, which are popular now, are very excellent system to receive the visible light. 

By using the visible light for the data transmission, many problems related to radio and infrared communications are solved. The visible light communication has characteristics to be ubiquitous, transmitted at ultra high speed and harmless for human body and electronic devices, compared to those by radio and infrared communications.

We wish to explore the possibilities laid down by VLC. In particular we want to evaluate the capacity of LEDs to transmit data at relatively high speed and over a short range. 

\subsection{Project Goal}
% summarize what your design project is to achieve. It can be   general and non-technical but should describe “the big picture” elaborately. Basically, you briefly explain what the solution is and what is unique about your solution.

Our project aims to achieve the the following objectives in order to make a satisfactory prototype
\begin{itemize}
\item Transmission speed should be 1 Mb/s
\item Blinking should not be detected by the naked eye
\item Link should work over a distance of 50 cm
\item Should transmit a file stored in a USB/SD card 
\end{itemize}

\subsection{Project Specifications}
\begin{itemize}
\item Customer Specifications: The following specifications will guarantee a finished product for consumer use
  \begin{itemize}
  \item Efficient, reliable and accurate transmission of the data at desired high speeds.
  \item Availability of ambient light, along with the simultaneous data transmission from the LEDs. Also, care will be taken that the flickering of the LEDs would be indiscernible, without compromising much on the illumination.  
  \end{itemize}
\item Technical Specifications: 
  \begin{itemize}
  \item Data Rate : The speed with which data can be transmitted from one device to another. (Transmitter to Receiver in this case). Our initial aim is to achieve a data rate of 1 Kbps and then increase and finally reach the desired speed of 1 Mbps.
  \item Bit Error Rate : It is the number of bit errors is the number of received bits of a data stream over a communication channel that have been altered due to noise, interference, distortion or bit synchronization errors. It is given by the formula, in On-Off Keying Modulation Scheme, as $$\textrm{OOK} : \textrm{BER} = \mathcal{N}(\sqrt{\textrm{SNR}})$$ We hope to achieve a BER of $10^{-4}$ and reduce it further if possible.
  \item Distance : We hope to achieve accurate data transmission upto a distance of 50 cm. If       we are done with the above objectives, we hope to increase the distance till 100 cm.  
  \end{itemize}
\end{itemize}

\section{Technical Design Description}
\subsection{Possible Solutions and Design Alternatives}
% In this section describe the possible solutions and evaluate them in terms their possible performance, availability of the resources, and other limitations. State the solution you would like to adopt with justification.
\subsection{System-level overview}
% Include a block diagram with functional description. A brief overview of what   your proposed device/solution is and how it will work. Include figures to illustrate the concept of your final product showing what it will look like, showing how it will work, and showing what it will do.
The following is the step-by-step description of the entire system right from acquisition of the data to its reception in a file on the computer.
\begin{itemize}
\item Data Input - The data will be first acquired by the microcontroller from the SD card. 
% \item Encoding - The Input Data will then be encoded using Hoffman Encoding (CHECK) to reduce the number of bits to transmit
\item Modulation - The encoded data will then be modulated and thus converted to a form which can be transmitted through the optical Channel via the LEDs (the Transmitter).
\item Channel - In visible light communication, it is pretty obvious that the data will be transmitted through an optical channel. Errors may arise due to interferences amongst symbols, or addition of noise or multipath fading
\item Frequency Offset Correction and Clock Synchronisation at the Receiver- The receiver will be more complicated in design than the transmitter because of this feature. Appropriate circuitry will be designed to correct the frequency offset and synchronise the clocks
\item Demodulation - Once the point mentioned above (5), gets implemented correctly, the modulated data will be demodulated to recover the encoded data
\item Decoding - The encoded data will be decoded to retrieve the original data intended to be sent
\item Saving the data - Finally the data which is received will then be saved in a file on the computer
\end{itemize}

\subsection{Performance Validation}
% Describe how you would validate your final design and prove that it meets the specifications you promised. You would demonstrate your successful project at the final Design Lab Demo.
The following measures will be noted will be used to values our performance by keeping a distance of 50 cm between the LED and the receiver:
\begin{itemize}
\item Time required to transmit given data from the LED to the receiver and thus the Data Rate
\item The Bit Error Rate (BER) - By comparing the input file with the received file
\end{itemize}

\section{Project Plan}
%This section is to provide 
% A listing of all tasks, planning, involving all the tasks and sub-tasks. A task can be defined as anything that takes your time.
% Examples of tasks possible are (but not limited to):  Embedded system design, sensor testing, analog module design and testing, mechanical design, PCB    design, power consumption estimation, components procurement, documentation, etc.

Tasks to be done:
\begin{itemize}
\item Reading Literature :  We would need to go through a basic survey regarding the basics of visible light communication, corresponding hardware, modulation techniques with pros and cons of each of them, and hence come with a viable solution incorporating the all the factors and the feasibility given the availability of components and time constraints. The sources mainly comprise a number of papers and the few books in the field
\item Circuit design :  Broadly speaking there would be two main parts to this, the transmitting circuit and the reception circuit. The transmission circuit would mainly involve the LED, a biasing circuit, and a switching circuit to generate the bit pattern amongst other basic elements. The reception circuit, which is relatively more difficult would mainly involve a photodetector,a circuit to nullify the ambient DC value and related basic elements.
\item Generation of bit stream :  Another task is about taking a file and converting it into a bit stream which can be used to drive or feed the switching circuit. The main part is first to get a serial output of the bit pattern of the file and then modulate the signal according to the modulation scheme chosen initially. There a number of things that can be carried out in addition to the basic process outlined here which have been described later in the possible solutions section.
\item Software implementation : This mainly would pertain to optimization of the implementation of the modulation scheme, error rate checking and correction and most importantly clock recovery and frequency offset correction hence synchronisation to improve the performance and the throughput of the system. This would mainly be done using a DSP board.
\item PCB  design : The design and development of a PCB for circuits of both the ends.
\item Integration and debugging:  The individual components would need to be debugged at all the stages and the all these components would need to be integrated to form a complete working system involving the debugging of the system after arranging them and finally packaging them.

\end{itemize}
\subsection{Work distribution}
\subsection{Gantt chart}
% Time line for execution including team members associated with each task   you have planned – A Gantt chart can be used for the purpose.

\section{Project Implementation}
% At this stage, you need to submit a BOM (Bill of Materials). Apart from the components decide your testing strategy - how to test, needed tools, precautions and feasibility Assessment (resources, risks).  What is the deliverable of the work and demo possible in reality?

\section{Deliverables}
% Spell out the project deliverables - what you would demonstrate during the evaluations
% (It is expected that for the first evaluation (1st week of Feb) roughly 30% of work is ready. You will have to demonstrate the subsystems that are ready. For the second evaluation (2nd week of March) 60 to 80% of your work should be over and you should be working on your final PCB, box etc. We expect a draft version of your final report submitted by the 1st week of April. For the final evaluation (April 10-14) you need to give a brief presentation (20 min) and a demo of your project.)

\begin{itemize}
\item By first week of February, we hope to get done with
  \begin{itemize}
  \item Understanding Literature
  \item Procurement of Components
  \item Finishing the Transmitter Circuit
  \item Getting Started with the Receiver 
  \end{itemize}

\item By second week of March, we hope to accomplish

  \begin{itemize}
  \item Finish Tackling the Problem of Clock Synchronization and Frequency offset
  \item Start integrating the components.
  \end{itemize}

\item By first week of April, we hope to accomplish
  \begin{itemize}
  \item After second evaluation, we get started with testing of the circuit. We will start off with 10 kbps, correct the errors which come in the way and in the end   hope to get the desired speed of 1Mbps. 
  \end{itemize}
\end{itemize}


\end{document}

